% Copyright (c) 2025 ZKLib Contributors. All rights reserved.
% Released under Apache 2.0 license as described in the file LICENSE.
% Authors: Poulami Das (Least Authority)

\section{Stir}

\subsection{Tools for Reed Solomon codes}
\subsubsection{Random linear combination as a proximity generator}\label{sec:proximity_gap}

\begin{theorem}\label{thm:proximity_gap}
\lean{proximity_gap}
\uses{def:distance_from_code,def:reed_solomon_code,def:list_close_codewords}
    Let $\code := \rscode[\field,\evaldomain,\degree]$ be a Reed Solomon code with rate $\rate:=\frac{\degree}{|\evaldomain|}$ and let $B^*(\rate):=\sqrt{\rate}$. For every $\distance \in (0,1 - B^*(\rate))$ and functions $f_0,\ldots,f_{m-1} : \evaldomain \to \field$, if
    \[
    \Pr_{r\leftarrow\field}\!\Bigl[
      \Delta\Bigl(\sum_{j=0}^{m-1} r^j\cdot {f_j},\rscode[\field,\evaldomain,\degree]\Bigr)
      \le \delta
    \Bigr]>\err^*(\degree,\rate,\distance,m),
    \]
    then there exists a subset $S \subseteq \evaldomain$ with $|S| \ge (1 - \delta)\cdot|L|$,
    and for every $i \in [m]$, there exists $u \in \rscode[\field,\evaldomain,\degree]$ such that $f_i(S) = u(S)$.
    
    \medskip
    
    \noindent
    Above, $\err^*(\degree,\rate,\distance,m)$ is defined as follows:
    \begin{itemize}
        \item if $\distance \in \left(0,\frac{1-\rate}{2}\right]$ then
            \[
                \err^*(\degree,\rate,\distance,m)=\frac{(m-1)\cdot \degree}{\rate\cdot|\field|}
            \]
        \item if $\distance \in \Bigl(\frac{1-\rate}{2}, 1-\sqrt{\rate}\Bigr)$ then
        \[
            \err^*(\degree,\rate,\distance,m)=\frac{(m-1)\cdot {\degree}^2}{|\field|\cdot{\Bigl(2\cdot\min\{1-\sqrt{\rate}-\distance,\frac{\sqrt{\rate}}{20}\}\Bigr)}^7}
        \]
    \end{itemize}
\end{theorem}

\subsubsection{Univariate Function Quotienting}\label{sec:quotienting}

In the following, we start by defining the \emph{quotient} of a univariate function.
\begin{definition}\label{def:quotient}
\lean{quotient}
    Let $f:\evaldomain\to\field$ be a function, $S\subseteq\field$ be a set, and $\mathsf{Ans},\mathsf{Fill}: S\rightarrow\field$ be functions. Let $\hat{\mathsf{Ans}}\in\field^{<|S|}[X]$ be the (unique) polynomial with $\hat{\mathsf{Ans}}(x)=\mathsf{Ans}(x)$ for every $x\in S$, and let $\hat{V}_S\in\field^{<|S|+1}[X]$ be the unique non-zero polynomial with $\hat{V}_S(x)=0$ for every $x\in S$.
    The \emph{quotient function} $\mathsf{Quotient}\bigl(f,S,\mathsf{Ans},\mathsf{Fill}\bigr): \evaldomain\to\field$
    is defined as follows:
\[
\forall x \in \evaldomain,\quad
\mathsf{Quotient}\bigl(f,S,\mathsf{Ans},\mathsf{Fill}\bigr)(x)
:=
\begin{cases}
\mathsf{Fill}(x)
& \text{if } x \in S \\[6pt]
\dfrac{f(x) - \hat{\mathsf{Ans}}(x)}{\hat{V}_S(x)}
& \text{otherwise}
\end{cases}
\]
\end{definition}

Next we define the polynomial quotient operator, which quotients a polynomial relative to its output on evaluation points. The polynomial quotient is a polynomial of lower degree.

\begin{definition}\label{def:poly_quotient}
\lean{polyQuotient}
    Let $\hat{f}\in\field^{<\degree}[X]$ be a polynomial and $S\subseteq\field$ be a set, let $\hat{V}_S\in\field^{<|S|+1}[X]$ be the unique non-zero polynomial with $\hat{V}_S(x)=0$ for every $x\in S$. The \emph{polynomial quotient} $\mathsf{PolyQuotient}(\hat{f},S)\in\field^{<d-|S|}[X]$ is defined as follows:
    \[
            \mathsf{PolyQuotient}(\hat{f},S)(X):=\frac{\hat{f}(X)-\hat{\mathsf{Ans}}(X)}{\hat{V}_S(X)}
    \]
where $\hat{Ans}\in\field^{<|S|}[X]$ is the unique non-zero polynomial with $\hat{Ans}(x)=\hat{f}(x)$ for every $x \in S$.
\end{definition}

The following lemma, implicit in prior works, shows that if the function is ``quotiented by the wrong value'', then its quotient is far from low-degree.

\begin{lemma}\label{lemma:quotienting}
\lean{quotienting}
\leanok
\uses{def:quotient,def:reed_solomon_code,def:distance_from_code,def:list_close_codewords}
    Let $f:\evaldomain\rightarrow\field$ be a function, $\degree\in\N$ be the degree parameter, $\distance\in(0,1)$
    be a distance parameter, $S\subseteq\field$ be a set with $|S|<\degree$, and $\mathsf{Ans},\mathsf{Fill}:S\rightarrow\field$ are functions. Suppose that for every $u\in\listcode(f,\degree,\distance)$ there exists $x\in S$ with $\hat{u}(x)\neq\mathsf{Ans}(x)$. Then 
    \[
            \Delta(\mathsf{Quotient}(f,S,\mathsf{Ans},\mathsf{Fill}),\rscode[\field,\evaldomain,\degree-|S|])+\frac{|T|}{|\evaldomain|}>\distance,
    \]
    where $T:=\{x\in\evaldomain\cap S: \hat{\mathsf{Ans}}(x)\neq f(x)\}$.
\end{lemma}

\subsubsection{Out of domain sampling}\label{sec:out_of_domain_smpl}
\begin{lemma}\label{lemma:out_of_domain_smpl}
\lean{out_of_dom_smpl_1, out_of_dom_smpl_2}
\uses{def:reed_solomon_code,def:list_decodable,def:list_close_codewords}
    Let $f:\evaldomain\rightarrow\field$ be a function, $\degree\in\N$ be a degree parameter, $s\in\N$ be a repetition parameter, and $\distance\in[0,1]$ be a distance parameter. If $\rscode[\field,\evaldomain,\degree]$ be $(\degree,l)$-list decodable then
    \begin{align*}
    \Pr_{r_1,\ldots,r_s\gets\field\setminus\evaldomain}[\exists\text{ distinct }u,u'\in\listcode(f,\degree,\distance):\forall i\in[s],\hat{u}(r_i)=\hat{u}'(r_i)] & \leq \binom{l}{2}\cdot{\Big(\frac{\degree-1}{|\field|-|\evaldomain|}\Big)}^s \\
    & \leq \Big(\frac{l^2}{2}\Big)\cdot{\Big(\frac{\degree}{|\field|-|\evaldomain|}\Big)}^s
    \end{align*}
\end{lemma}

\subsubsection{Folding univariate functions}\label{sec:folding_uf}
STIR relies on $k$-wise folding of functions and polynomials - this is similar to prior works, although presented in a slightly different form. As shown below, folding a function preserves proximity from the Reed-Solomon code with high probability. The folding operator is based on the following fact, decomposing univariate polynomials into bivariate ones.

\begin{lemma}\label{fact:poly_folding} 
\lean{exists_unique_bivariate,degree_bound_bivariate}
Given a polynomial $\hat{q}\in\field[X]$:
\begin{itemize}
        \item For every univariate polynomial $\hat{f}\in\field[X]$, there exists a unique bivariate polynomial $\hat{Q}\in\field[X,Y]$ with:
        \[
          \polydeg_X(\hat{Q}) := \left\lfloor \frac{\polydeg(\hat{f})}{\polydeg(\hat{q})} \right\rfloor,\quad \polydeg_Y(\hat{Q}) < \polydeg(\hat{q})
        \]
        such that $\hat{f}(Z)=\hat{Q}(\hat{q}(Z),Z)$. Moreover, $\hat{Q}$ can be computed efficiently given $\hat{f}$ and $\hat{q}$. Observe that if $\polydeg(\hat{f})<t\cdot\polydeg(\hat{q})$ then
        $\polydeg(\hat{Q})<t$.
        
        \item For every $\hat{Q}[X,Y]$ with $\polydeg_X(\hat{Q})<t$ and $\polydeg_Y(\hat{Q})<\polydeg(\hat{q})$, the polynomial $\hat{f}(Z)=\hat{Q}(\hat{q}(Z),Z)$ has degree $\polydeg(\hat{f})<t\cdot\polydeg(\hat{q})$.
\end{itemize}
\end{lemma}

Below, we define folding of a polynomial followed by folding of a function.
\begin{definition}\label{def:poly_folding}
\lean{polyFold}
\uses{fact:poly_folding}
    Given a polynomial $\hat{f}\in\field^{<\degree}[X]$, a folding parameter $k\in\N$ and $r\in\field$, we define a polynomial $\mathsf{PolyFold}(\hat{f},k,r)\in\field^{\degree/k}[X]$ as follows. Let $\hat{Q}[X,Y]$ be the bivariate polynomial derived from $\hat{f}$ using Fact~\ref{fact:poly_folding} with $\hat{q}(X):=X^k$. Then $\mathsf{PolyFold}(\hat{f},k,r)(X):=\hat{Q}(X,r)$.
\end{definition}

\begin{definition}\label{def:fn_folding}
\lean{fold}
Let $f:\evaldomain\rightarrow\field$ be a function, $k\in\N$ a folding parameter and $\alpha\in\field$. For every $x\in{\evaldomain}^k$, let $\hat{p}_x\in\field^{<k}[X]$ be the polynomial where $\hat{p}_x(y)=f(y)$ for every $y\in{\evaldomain}$ such that $y^k=x$. We define $\mathsf{Fold}(f,k,\alpha):\evaldomain\rightarrow\field$ as follows.
\[
    \mathsf{Fold}(f,k,\alpha):=\hat{p}_x(\alpha).
\]
In order to compute $\mathsf{Fold}(f,k,\alpha)(x)$ it suffices to interpolate the $k$ values $\{f(y):y\in\evaldomain\text{ s.t. }y^k=x\}$ into the polynomial $\hat{p}_x$ and evaluate this polynomial at $\alpha$.
\end{definition}

The following lemma shows that the distance of a function is preserved under folding. If a functions $f$ has distance $\distance$ to a Reed-Solomon code then, with high probability over the choice of folding randomness, its folding also has a distance of $\distance$ to the ``$k$-wise folded'' Reed-Solomon code.

\begin{lemma}\label{lemma:folding}
\lean{folding}
\uses{def:reed_solomon_code,def:fn_folding,def:dist_from_code}
    For every function $f:\evaldomain\rightarrow\field$, degree parameter $\degree\in\N$, folding parameter $k\in\N$, distance parameter $\distance\in(0,\min\{{\Delta(\mathsf{Fold}[f,k,r^{\mathsf{fold}}],\rscode[\field,{\evaldomain}^k, \degree/k]),1-{\mathsf{B}}^*(\rate)}\})$, letting $\rate:=\frac{\degree}{|\evaldomain|}$,
    \[
        \Pr_{r^{\mathsf{fold}}\gets\field}[\Delta(\mathsf{Fold}[f,k,r^{\mathsf{fold}}],\rscode[\field,{\evaldomain}^k, \degree/k])<\distance]>\err^*(\degree/k,\rate,\distance,k).
    \]
    Above, ${\mathsf{B}}^*$ and ${\err}^*$ are the proximity bound and error (respectively) described in Section~\ref{sec:proximity_gap}.
\end{lemma}

\subsubsection{Combine functions of varying degrees}\label{sec:combine}
We show a new method for combining functions of varying degrees with minimal proximity require- ments using geometric sums. We begin by recalling a fact about geometric sums.

\begin{lemma}\label{fact:geometric_sum}
\lean{geometric_sum_units}
    Let $\field$ be a field, $r\in\field$ be a field element, $a\in\N$ be a natural number. Then
    \[
        \sum_{i=0}^{a}r^i:=
        \begin{cases}
        \Big(\frac{1-r^{a+1}}{1-r}\Big) & r\neq 1 \\
        a+1 & r=1
        \end{cases}
    \]
\end{lemma}

\begin{definition}\label{def:combine}
\lean{combineInterm,combineFinal}
    Given target degree $\degree^*\in\N$, shifting parameter $r\in\field$, functions $f_0,\ldots,f_{m-1}:\evaldomain\rightarrow\field$, and degrees $0\leq \degree_0,\ldots,\degree_{m-1}\leq {\degree}^*$, we define $\combine(\degree^*,r,(f_0,\degree_0),\ldots,(f_{m-1},\degree_{m-1})):\evaldomain\rightarrow\field$ as follows:
    \begin{align*}
        \combine({\degree}^*,r,(f_0,\degree_0),\ldots,(f_{m-1},\degree_{m-1}))(x)&:=\sum_{i=0}^{m-1}r_i\cdot f_i(x)\cdot \Big(\sum_{l=0}^{\degree^*-\degree_i}{(r\cdot x)}^l\Big) \\
        &= 
        \begin{cases}
            \sum_{i=0}^{m-1}r_i\cdot f_i(x)\cdot \Big(\frac{1-{(xr)}^{\degree^*-\degree_i+1}}{1-xr}\Big) & x\cdot r\neq 1\\
            \sum_{i=0}^{m-1}r_i\cdot f_i(x)\cdot (\degree^*-\degree_i+1) & x\cdot r= 1
        \end{cases}
    \end{align*}
Above, $r_i:=r^{i-1+\sum_{j<i}(\degree^*-\degree_j)}$.
\end{definition}

\begin{definition}\label{def:deg_corr}
\lean{degreeCorrInterm,degreeCorrFinal}
\uses{def:combine,def:reed_solomon_code,def:dist_from_code}
    Given target degree $\degree^*\in\N$, shifting parameter $r\in\field$, function $f:\evaldomain\rightarrow\field$, and degree $0\leq\degree\leq\degree^*$, we define $\degcorr(\degree^*,r,f,\degree)$ as follows.
    \[
        \degcorr(\degree^*,r,f,\degree)(x):=f(x)\cdot\Bigg(\sum_{l=0}^{\degree^*-\degree}{(r\cdot x)}^l\Bigg)=
        \begin{cases}
            f(x)\cdot\frac{1-{(xr)}^{\degree^*-\degree_i+1}}{1-xr} & x\cdot r\neq 1\\
            f(x)\cdot (\degree^*-\degree_i+1) & x\cdot r = 1
        \end{cases}
    \]
(Observe that $\degcorr(\degree^*,r,f,\degree)=\combine(\degree^*,r,(f,\degree))$.)
\end{definition}

Below it is shown that combining multiple polynomials of varying degrees can be done as long as the proximity error is bounded by $(\min{\{1-{\mathsf{B}}^*(\rate),1-\rate-1/|\evaldomain|\}})$.

\begin{lemma}\label{lemma:combine}
\lean{combine}
\uses{def:reed_solomon_code,def:combine,def:deg_corr,thm:proximity_gap}
    Let $\degree^*$ be a target degree, $f_0,\ldots,f_{m-1}:\evaldomain\rightarrow\field$ be functions, $0\leq \degree_0,\ldots,\degree_{m-1}\leq \degree^*$ be degrees, $\distance\in\min{\{1-{\mathsf{B}}^*(\rate),1-\rate-1/|\evaldomain|\}}$ be a distance parameter, where $\rate=\degree^*/|\evaldomain|$. If
    \[
        \Pr_{r\gets\field}[\Delta(\combine(\degree^*,r,(f_0,\degree_0),\ldots,(f_{m-1},\degree_{m-1})),\rscode[\field,\evaldomain,\degree^*])]>\err^*(\degree^*,\rate,\distance,m\cdot(\degree^*+1)-\sum_{i=0}^{m-1}\degree_i),
    \] 
    then there exists $S\subseteq \evaldomain$ with $|S|\geq(1-\distance)\cdot|\evaldomain|$, and
    \[
        \forall i\in[m-1],\exists u\in\rscode[\field,\evaldomain,\degree_i], f_i(S)=u(S).
    \]
    Note that this implies $\Delta(f_i,\rscode[\field,\evaldomain,\degree_i])<\distance$ for every $i$. Above, ${\mathsf{B}}^*$ and ${\err}^*$ are the proximity bound and error (respectively) described in Section~\ref{sec:proximity_gap}.
\end{lemma}

\subsection{Stir Main theorems}

\begin{theorem}[STIR Main Theorem]\label{thm:stir}
\lean{stir_main}
\uses{def:reed_solomon_code,lemma:rnd_by_rnd_soundness}
    Consider the following ingrediants:
    \begin{itemize}
        \item A security parameter $\lambda\in\N$.
        \item A Reed-Solomon code $\rscode[\field,\evaldomain,\degree]$ with $\rate:=\frac{\degree}{|\evaldomain|}$ where $\degree$ is a power of $2$, and $\evaldomain$ is a smooth domain.
        \item A proximity parameter $\distance\in(0,1-1.05\cdot\sqrt{\rate})$.
        \item A folding parameter $k\in\N$ that is power of $2$ with $k\geq 4$.
    \end{itemize}
If $|\field|=\Omega(\frac{\lambda\cdot2^\lambda\cdot\degree^2\cdot{|\evaldomain|}^2}{\log(1/\rate)})$, there is a public-coin IOPP for $\rscode[\field,\evaldomain,\degree]$ with the following parameters:
\begin{itemize}
    \item Round-by-round soundness error $2^{-\lambda}$.
    \item Round complexity: $M:=O(\log_k{\degree})$.
    \item Proof length: $|\evaldomain|+O_k(\log{\degree})$.
    \item Query complexity to the input: $\frac{\lambda}{-\log{(1-\distance)}}$.
    \item Query complexity to the proof strings: $O_k(\log{\degree}+\lambda\cdot\log{\Big(\frac{\log{\degree}}{\log{1/\rate}}\Big)})$.
\end{itemize}
\end{theorem}

\begin{lemma}\label{lemma:rnd_by_rnd_soundness}
\lean{stir_rbr_soundness}
\uses{def:reed_solomon_code,def:list_decodability,lemma:folding,lemma:out_of_domain_smpl,lemma:quotienting,lemma:combine}
    Consider $(\field,M,\degree,k_0,\ldots,k_M,\evaldomain_0,\ldots,\evaldomain_M,t_0,\ldots,t_M)$ and for every $i\in\{0,\ldots,M\}$, let $\degree_i:=\frac{\degree}{\prod_{j<i}k^j}$ and $\rate_i:=\degree_i/|\evaldomain_i|$. For every $f\notin\rscode[\field,\evaldomain_0,\degree_0]$ and every $\distance_0,\ldots,\distance_M$ where
    \begin{itemize}
        \item $\distance_0\in(0,\Delta(f,\rscode[\field,\evaldomain_0,\degree_0])]\cap(0,1-{\mathsf{B}}^*(\rate_0))$
        \item for every $0<i\leq M$: $\distance_i\in(0,\min{\{1-\rate_i-\frac{1}{|\evaldomain_i|},1-{\mathsf{B}^*(\rate_i)}\}})$, and
        \item for every $0<i\leq M$: $\rscode[\field,\evaldomain_i,\degree_i]$ is $(\distance_i,l_i)$-list decodable,
    \end{itemize}
    There exists an IOPP that has round-by-round soundness error $(\epsilon^{\mathsf{fold}},\epsilon^{\mathsf{out}}_1,\epsilon^{\mathsf{shift}}_1,\ldots,\epsilon^{\mathsf{out}}_M,\epsilon^{\mathsf{shift}}_M,\epsilon^{\mathsf{fin}})$ where:
    \begin{itemize}
        \item $\epsilon^{\mathsf{fold}}\leq\err^*(\degree_0/k_0,\rate_0,\distance_0,k_0)$.
        \item $\epsilon^{\mathsf{out}}_i\leq\frac{l^2_i}{2}\cdot{\big(\frac{\degree_i}{|\field|-|\evaldomain_i|}\big)}^s$
        \item $\epsilon^{\mathsf{shift}}_i\leq {(1-\distance_{i-1})}^{t_{i-1}}+\err^*(\degree_i,\rate_i,\distance_i,t_{i-1}+s)+\err^*(\degree_i/k_i,\rate_i,\distance_i,k_i)$.
        \item $\epsilon^{\mathsf{fin}}\leq{(1-\distance_M)}^{t_M}$.
    \end{itemize}
    Above, ${\mathsf{B}}^*$ and ${\err}^*$ are the proximity bound and error (respectively) described in Section~\ref{sec:proximity_gap}.
\end{lemma}

