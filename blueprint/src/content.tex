% In this file you should put the actual content of the blueprint.
% It will be used both by the web and the print version.
% It should *not* include the \begin{document}
%
% If you want to split the blueprint content into several files then
% the current file can be a simple sequence of \input. Otherwise It
% can start with a \section or \chapter for instance.

\chapter{Introduction}

The goal of this project is to formalize Succinct Non-Interactive Arguments of Knowledge (SNARKs) in Lean. Our focus is on SNARKs based on Interactive Oracle Proofs (IOPs). We plan to build a general framework for IOP-based SNARKs that can state specifications of the protocols and prove their security properties in a clean and modular way.

\chapter{Oracle Reductions}

\section{Definitions}

\begin{definition}[Interactive Protocol]
    \label{def:interactive_protocol}
    \lean{ProtocolSpec}
    An \emph{$n$-round interactive protocol} between two parties $P, V$ is a sequence of messages $c_0, m_0, \dots, c_n, m_n$ where:
    \begin{itemize}
        \item $c_i$ is a challenge sent by $V$ to $P$ in the $i$-th round.
        \item $m_i$ is a message sent by $P$ to $V$ in the $i$-th round.
    \end{itemize}
    Each message $m_i$ and challenge $c_i$ may be of different types. We bundle them all together as a \verb|ProtocolSpec| structure.
\end{definition}

\begin{definition}[Oracle Reduction]\label{def:oracle_reduction}
    \lean{Protocol}
    An \emph{(interactive) oracle reduction} is an interactive protocol with a prover and a verifier.
\end{definition}

\begin{definition}[Completeness]
    \label{def:completeness}
    \lean{completeness}
\end{definition}

\section{Composition}

\chapter{Commitment Schemes}

\section{Definitions}

\section{Composition}



\chapter{Proof Systems}

\section{The Sum-Check Protocol}

\section{The Spartan Protocol}

\section{The Ligero Polynomial Commitment Scheme}


\chapter{Supporting Results}

\section{Polynomials}

\section{Coding Theory}


\chapter{References}

