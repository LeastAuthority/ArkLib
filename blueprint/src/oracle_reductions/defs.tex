\section{Definitions}

\iffalse

New notes on what I envision for IORs now:

1. It should follow as closely as possible a program logic. This means that an oracle reduction itself should be thought of as a monad in some way. Perhaps "pure" is a protocol with a single round, and "bind" is concatenating protocols together.

2. Given this approach, we should approach the (oracle) statement & witness as the context. So we will get a list of public inputs, a list of private inputs, and a list of oracles (available as inputs to the prover, but as oracles to the verifier). This context will be append-only as the reduction unfolds (via appending the transcript) so that we can use PL language like

" Ctx |- P, V "

3. Then security properties of this oracle reduction language are seen as just Hoare triples. The input & output relations are just the pre & post conditions of the Hoare triple. The Hoare triple will have different semantics depending on the security property we consider.

In the form of Dijkstra monads, this means that we have different specification monads for different security properties.

4. Another benefit of this approach is that we can freely use augmented spec monads with "ghost" states, to express reductions of the form "apply X protocol to Y values derived from the context".

5. Of course, sequential composition is immediate from the fact that this is a programming language with a small-step semantics (i.e. what each message-passing step does). To show that completeness compose, this seems to follow directly from sequencing / bind rules in Hoare logic.

6. In general, it is still not clear how to state knowledge soundness in this framework. Perhaps think of verifier + extractor as a single entity, then inject in the prover as nothing other than non-deterministic choice of the prover's messages (we work in the unbounded prover model for now).

7. What about Merkle trees & queries to random oracles? My old view is that we can "lay them out" like a partial heap, where the values are defined on the go. So then we just get an augmented context, and the extractor can see this context. The semantics are probabilistic, of course.

8. Another big question is how to model the "substitution" of one oracle for another, followed by a proof / reduction that the new oracle faithfully simulates the old one (up to some error).

\fi

In this section, we give the basic definitions of a public-coin interactive oracle reduction
(henceforth called an oracle reduction or IOR). In particular, we will define its building blocks,
and various security properties.

\subsection{Format}

An oracle reduction is an interactive protocol between two parties, a \emph{prover} $\mathcal{P}$
and a \emph{verifier} $\mathcal{V}$. Its format is as follows:
\begin{enumerate}
    \item The protocol structure is fixed and defined by a given \emph{type signature}, which
    describes in each round which party sends a message to the other, and the type of that message.
    \item All messages from $\mathcal{V}$ are chosen uniformly at random (more generally, from some
    fixed probability distribution).
    \item The messages sent from the prover may either: 1) be seen directly by the verifier, or 2)
    only available to a verifier through an \emph{oracle interface} (which specifies the type for
    the query and response, and the oracle's behavior given the underlying message).
    \item The prover and verifier has access to some inputs at the beginning of the protocol. These inputs are classified as follows:
    \begin{itemize}
        \item \emph{Public inputs}: available to both parties;
        \item \emph{Private inputs} (or \emph{witness}): available only to the prover;
        \item \emph{Oracle inputs}: the underlying data is available to the prover, but it's only
        exposed as an oracle to the verifier.
        \item \emph{Shared oracle}: the oracle is available to both parties via an interface, and it
        may be randomized.
    \end{itemize}
\end{enumerate}

We collect all the public inputs, private inputs, and oracles into a \emph{context}.

\begin{definition}[Context]\label{def:context}
    In an oracle reduction, its \emph{context} consists of a list of public inputs, a list of witness inputs, a list of oracle inputs, and a shared oracle (possibly represented as a list of lazily sampled query-response pairs). These inputs have the expected visibility.
\end{definition}

We imagine the context as \emph{append-only}, as we add new messages from the protocol execution.

\begin{definition}[Type Signature of an Oracle Reduction]
    \label{def:oracle_reduction_type_signature}
    An \emph{$n$-message oracle reduction} between two parties $\mathcal{P}$ and $\mathcal{V}$
    consists of a sequence of messages $m_0, \dots, m_n$, where each message $m_i$ (of a given
    type) is associated with a \emph{direction} (to $\mathcal{P}$ or to $\mathcal{V}$), and a
    message visibility (public or oracle) if coming from $\mathcal{P}$.
    
    % The type signature of an oracle reduction is given by an object of type
    %     \verb|ProtocolSpec (n : Nat) :=
    % Fin n → (Direction × Visibility × Type)|.

    % This definition generalizes similar ones in the literature in that we do not fix a particular protocol flow (e.g. alternating between prover's and verifier's messages). This is meant to capture all protocols in the most generality, and have some benefits in composition.
    % \lean{ProtocolSpec}
\end{definition}

% In the interactive protocols we consider, both parties $P$ and $V$ may have access to a shared
% oracle $O$. An interactive protocol becomes an \emph{interactive (oracle) reduction} if its
% execution reduces an input relation $R_{\mathsf{in}}$ to an output relation $R_{\mathsf{out}}$. Here
% a relation is just a function $\mathsf{IsValid}: \mathsf{Statement} \times \mathsf{Witness} \to
% \mathsf{Bool}$, for some types \verb|Statement| and \verb|Witness|. We do not concern ourselves with
% the running time of $\mathsf{IsValid}$ in this project (though future extensions may prove that
% relations can be decided in polynomial time, for a suitable model of computation).

% \begin{remark}[Design Decision]
%     We do not enforce a particular interaction flow in the definition of an interactive (oracle) reduction. This is done so that we can capture all protocols in the most generality. Also, we want to allow the prover to send multiple messages in a row, since each message may have a different oracle representation (for instance, in the Plonk protocol, the prover's first message is a 3-tuple of polynomial commitments.)
% \end{remark}

\begin{definition}[Type Signature of a Prover]
    \label{def:prover_type_signature}
    A prover $\mathcal{P}$ in an oracle reduction, given a context, is a stateful oracle computation
    that at each step of the protocol, either takes in a new message from the verifier, or sends a
    new message to the verifier.
    \lean{Prover}
\end{definition}

Our modeling of oracle reductions only consider \emph{public-coin} verifiers; that is, verifiers who
only outputs uniformly random challenges drawn from the (finite) types, and uses no other
randomness. Because of this fixed functionality, we can bake the verifier's behavior in the
interaction phase directly into the protocol execution semantics.

After the interaction phase, the verifier may then run some verification procedure to check the
validity of the prover's responses. In this procedure, the verifier gets access to the public part
of the context, and oracle access to either the shared oracle, or the oracle inputs.
% This procedure differs depending on whether the verifier has
% full access, or only oracle access, to the prover's messages. Note that there is no difference on
% the prover side whether the protocol is an \emph{interactive oracle reduction (IOR)} or simply an
% \emph{interactive reduction (IR)}.

\begin{definition}[Type Signature of a Verifier]
    \label{def:verifier_type_signature}
    A verifier $\mathcal{V}$ in an oracle reduction is an oracle computation that may perform a
    series of checks (i.e. `Bool`-valued, or `Option Unit`) on the given context.
    \lean{Verifier}
\end{definition}

% \begin{definition}[Type Signature of an Oracle Verifier]
%     \label{def:oracle_verifier_type_signature}
%     \lean{OracleVerifier}
% \end{definition}

An oracle reduction then consists of a type signature for the interaction, and a pair of prover and
verifier for that type signature.

% \begin{definition}[Interactive Reduction]
%     \label{def:interactive_reduction}
%     \lean{Reduction}
%     An interactive reduction is a combination of a type signature \verb|ProtocolSpec|, a prover for \verb|ProtocolSpec|, and a verifier for \verb|ProtocolSpec|.
% \end{definition}

\begin{definition}[Interactive Oracle Reduction]
    \label{def:interactive_oracle_reduction}
    \lean{OracleReduction}
    An interactive oracle reduction is a combination of a type signature \verb|ProtocolSpec|, a prover for \verb|ProtocolSpec|, and an oracle verifier for \verb|ProtocolSpec|.
\end{definition}

We now define what it means to execute an oracle reduction. This is essentially achieved by first
executing the prover, interspersed with oracle queries to get the verifier's challenges (these will
be given uniform random probability semantics later on), and then executing the verifier's checks.
Any message exchanged in the protocol will be added to the context. We may also log information
about the execution, such as the log of oracle queries for the shared oracles, for analysis purposes
(i.e. feeding information into the extractor).

\begin{definition}[Execution of an Oracle Reduction]
    \label{def:oracle_reduction_execution}
    \lean{OracleReduction.run}
\end{definition}

\begin{remark}[More efficient representation of oracle reductions]
    The presentation of oracle reductions as protocols on an append-only context is useful for
    reasoning, but it does not lead to the most efficient implementation for the prover and
    verifier. In particular, the prover cannot keep intermediate state, and thus needs to recompute
    everything from scratch for each new message.

    To fix this mismatch, we will also define a stateful variant of the prover, and define a notion
    of observational equivalence between the stateless and stateful reductions.
\end{remark}

\subsection{Security properties}

We can now define properties of interactive reductions. The two main properties we consider in this
project are completeness and various notions of soundness. We will cover zero-knowledge at a later
stage.

First, for completeness, this is essentially probabilistic Hoare-style conditions on the execution
of the oracle reduction (with the honest prover and verifier). In other words, given a predicate on
the initial context, and a predicate on the final context, we require that if the initial predicate
holds, then the final predicate holds with high probability (except for some \emph{completeness}
error).

\begin{definition}[Completeness]
    \label{def:completeness}
    \lean{Reduction.completeness}
\end{definition}

Almost all oracle reductions we consider actually satisfy \emph{perfect completeness}, which
simplifies the proof obligation. In particular, this means we only need to show that no matter what challenges are chosen, the verifier will always accept given messages from the honest prover.

For soundness, we need to consider different notions. These notions differ in two main aspects:
\begin{itemize}
    \item Whether we consider the plain soundness, or knowledge soundness. The latter relies on the
    notion of an \emph{extractor}.
    \item Whether we consider plain, state-restoration, round-by-round, or rewinding notion of
    soundness.
\end{itemize}

We note that state-restoration knowledge soundness is necessary for the security of the SNARK
protocol obtained from the oracle reduction after composing with a commitment scheme and applying
the Fiat-Shamir transform. It in turn is implied by either round-by-round knowledge soundness, or
special soundness (via rewinding). At the moment, we only care about non-rewinding soundness, so mostly we will care about round-by-round knowledge soundness.

\begin{definition}[Soundness]
    \label{def:soundness}
    \lean{Reduction.soundness}
\end{definition}

A (straightline) extractor for knowledge soundness is a deterministic algorithm that takes in the output public context after executing the oracle reduction, the side information (i.e. log of oracle queries from the malicious prover) observed during execution, and outputs the witness for the input context.

Note that since we assume the context is append-only, and we append only the public (or oracle)
messages obtained during protocol execution, it follows that the witness stays the same throughout
the execution.

\begin{definition}[Knowledge Soundness]
    \label{def:knowledge_soundness}
    \lean{Reduction.knowledgeSoundness}
\end{definition}

To define round-by-round (knowledge) soundness, we need to define the notion of a \emph{state function}. This is a (possibly inefficient) function $\mathsf{StateF}$ that, for every challenge sent by the verifier, takes in the transcript of the protocol so far and outputs whether the state is doomed or not. Roughly speaking, the requirement of round-by-round soundness is that, for any (possibly malicious) prover $P$, if the state function outputs that the state is doomed on some partial transcript of the protocol, then the verifier will reject with high probability.

\begin{definition}[State Function]
    \label{def:state_function}
    \lean{Reduction.StateFunction}
\end{definition}

\begin{definition}[Round-by-Round Soundness]
    \label{def:round_by_round_soundness}
    \lean{Reduction.rbrSoundness}
\end{definition}

\begin{definition}[Round-by-Round Knowledge Soundness]
    \label{def:round_by_round_knowledge_soundness}
    \lean{Reduction.rbrKnowledgeSoundness}
\end{definition}